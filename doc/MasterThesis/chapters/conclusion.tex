% !TEX root = ../main.tex
\chapter{Related work}
Rusts design principles strongly include memory safety and other safety properties.
And there is an effort in the language community to formalize and proof these properties.
A core part of Rusts memory management was modeled in a formalism named Patina by Reed\cite{reed2015patina} in 2015.
Patinas statements satisfy memory safety properties like initialization before use or aliasing bounds for mutable memory.
$\lambda_{Rust}$ by Jung et al.\cite{Jung:2017:RSF:3177123.3158154} extends safety statements to unsafe code (where the Rust borrow checker does not enforce its strong rules) and was verified to hold the formulated safety guarantees.
Recently Jung et al. published another approach to minimize undefined behavior (where compiled code can be unpredictable due to different compiler implementations) in unsafe code.
These are important approaches to proof the guarantees that the language claims to give.
However, guarantees outside these boundaries have to be verified by other means.
Besides regular methods like unit and integration tests there is a model checking effort by Toman et al. \cite{toman2015crust} to give further memory safety guarantees especially on unsafe code.

Despite Petri-Net models are seemingly not used for verification of traditional programming languages, there was some effort to model general concurrent programs with the Basic Petri Net Programming Notation B(PN)$^2$ by Best et al. \cite{Best1993BPN2A}.
They used multilabled nets (M-nets)\cite{best1995class}, a class of high-level Petri-Nets for their approach.
Fleischhack et al. extended B(PN)$^2$ with procedures -- including recursion\cite{fleischhack1997petri}.
There is also research on Petri-Net semantics for description languages like the commonly used Specification and Description Language (SDL)\cite{fleischhack1998compositional} (also based on M-nets) or the Business Process Execution Language for Web Services (BPEL)\cite{stahl2005petri}\cite{lohmann2007feature}.
Both are used to verify properties of processes that are formulated in their description language.
Also, the $\pi$-calculus is backed by a Petri-Net semantics\cite{busi1995petri} based on low-level Petri-Nets with inhibitor arcs (inhibitor arcs require the connected preplace to be empty to activate a transition).

\chapter{Conclusion}
\label{conclusion}
The main goal of this work was finding a mapping from Rust programs to Petri-Nets.
A translated net then was intended to be used in a model checker to find deadlocks.

To reach that goal we searched for a suitable representation for Rust programs and developed a set of rules to translate that representation into Petri-Nets.
We did this for the basic components and constructed a complete model out of that components.
Because some important flow related information -- like blocking execution -- is hard to detect with our approach, we also added an emulation for Rust mutex locks.
And finally we tested if a simple test program can be translated and verified with a model checker to find the expected deadlock.

An analysis of our translation showed that our data model is very abstract and probably can be further improved.
However, the model of program flow seems to be close to the execution semantics of Rust programs.
Our test showed the expected behavior, but complex programs where not tested because the implementation does not cover all necessary features.
Yet, the general approach seems to be applicable and can be refined further to deal with complex scenarios.
