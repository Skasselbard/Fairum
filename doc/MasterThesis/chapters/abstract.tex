% !TEX root = ../main.tex
% TODO: Schau dir nochmal ein paar Paper abstracts an.
% Du solltest vorher noch einen Satz zur Motivation schreiben, und vor den letzten Satz noch einen Satz zu Ergebnissen schreiben.
\section*{Eine Petrinetzsemantik für Rust}
Es wird ein allgemeiner Ansatz gezeigt Rustprogramme in ein Petrinetzmodel zu überführen.
Die Übersetzung eines Beispielprogramms wird als Eingabe in einem Model-Checker verwendet um ein Deadlock zu finden, der durch mehrfaches blockieren eines Mutex verursacht wir.
Anschließend werden einige Vorschläge zur Verbesserung der gezeigten Übersetzung diskutiert.

\section*{A Petri-Net semantics for Rust}
We show a general approach to translate a Rust program into a Petri-Net model.
An example program is used as input for a model checker to find a contained deadlock.
The deadlock is caused by locking a mutex multiple times.
In the last part of this work we discuss how our approach can be improved further.

\vfill

\begin{tabular}{ll}
	\bfseries Betreuer: & \parbox[t]{10cm}{\betreuer }\vspace{5mm} \\
	\bfseries Tag der Ausgabe: & 27.09.2019 \\
	\bfseries Tag der Abgabe: & 13.03.2020 \\
\end{tabular}
