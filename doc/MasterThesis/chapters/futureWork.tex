% !TEX root = ../main.tex
\chapter{Future Work}
\label{future}
Although our approach produces Petri-Nets that are close to the Rust semantics, there is a lot of space to improve.
First, necessary flow related properties have to be modeled or emulated.
On the one hand a mechanism for splitting execution flow has to be integrated.
Primarily that means appropriate handling of threads, most likely by emulating the functionality of spawning and joining them.
In a Petri-Net that maps simply to a transition that produces a token on two separate places or consuming from two places respectively.
On the other hand, the model for guarding critical sections has to be refined further.
While the Petri-Net representation here is simple, a sound concept for the Rust side has to be found.
Emulation of mutexes probably already catches a lot of scenarios but others can be found where this not suffice.
For example low-level \textit{no\_std} environments where the mutexes from the standard library cannot be used.
Additionally, the current implementation actively marks locals to distinguish between mutex instances while this probably can be inferred.

The currently used data model can be improved as well.
Data that moves between locals or moves into or out of structures is currently modeled independently for every local, which, in turn, masks its semantic connection.
The movement might be modeled in a Petri-Net by separating the data from the local state.
A move then indicates that the previous local cannot access the data anymore, quite similar to the Rust ownership model.
If this can be done close to the Rust semantics, it might already fix the problem with marking mutexes.

Given a solid model with a reasonable control flow emulation, more complex scenarios should be tested.
This could include artificial ones like Dijkstras dining philosophers \cite{dijkstra1971hierarchical} or real life programs.
Larger test cases would be critical to decide if the verification process is efficient enough to be used in authentic use cases.
Test analysis would also improve from more sophisticated verification results.
Currently, the witness path is only a chain of transition ids.
But the MIR stores source-file-location-information that could be linked with the corresponding Petri-Net nodes.
It is likely that this information can be used to map the witness path to the original program source code.
This would improve usability a lot.

Furthermore, a graph representation of the MIR might help in the development process for MIR generation.
For example the missing storage statements we talked about in chapter \ref{results} left the initialized and dead place unconnected in the Petri-Net (which was excluded from the image).
This is a graph property that can be verified and might indicate a bug.
If more graph properties should be met by the MIR graph, they could be included into a test case to improve the compiler development process.

And finally, lifting the model to high-level Petri-Nets could be a solution to some intrinsic shortcomings (like the recursion restriction) and open the door to data dependent verification properties.
