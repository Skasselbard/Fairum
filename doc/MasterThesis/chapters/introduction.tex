% !TEX root = ../main.tex
\chapter{Introduction}
\label{introduction}

Here is a simple Rust\cite{klabnik2018rust} program that stops execution before it can terminate successfully:

\lstset{language=Rust,caption={A deadlock!},label=deadlock_program, frame=none, stepnumber=5, backgroundcolor=\color{verylightgray}}
\begin{lstlisting}
use std::sync::{Arc, Mutex};

pub fn main() {
    let data = Arc::new(Mutex::new(0));
    let _d1 = data.lock();
    let _d2 = data.lock();
}
\end{lstlisting}
The reason is a deadlock caused by locking a mutex twice without releasing it in the meantime.
Rust is a language that is highly concerned with memory safety and concurrency\cite{Matsakis:2014:RL:2692956.2663188} but the detection of deadlocks is explicitly excluded from the design\cite[Chapter 8.1]{nomicon}(for good reasons).
Nevertheless, it could be invaluable to detect such a situation automatically.
A proven method to do so is model checking\cite{baier2008principles} there we check a model of our program against certain properties.

In this work we will:
\begin{itemize}
    \item develop a Petri-Net\cite{petri1962kommunikation} semantics for Rust programs to serve as a model in chapter \ref{approach} (especially chapter \ref{app_trans}) ,
    \item find a mutex semantics for our model in chapter \ref{emulation}
    \item detect the deadlock from listing \ref{deadlock_program} with a model checker in chapter \ref{results},
    \item investigate the result and the shortcomings of our approach in chapter \ref{eval}
    \item and show a list of improvements that could lift our approach to be usable in realistic use cases in chapter \ref{future}
\end{itemize}
But first, we take a look on some important concepts in the next chapter.