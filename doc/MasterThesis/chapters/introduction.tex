% !TEX root = ../main.tex
\chapter{Introduction}
\label{introduction}
\section{Motivation}
High quality work is the desire of a professional;
Consequently, high quality software is the desire of a professional software engineer.

However, achieving high quality is an ambiguous task, since it is rather context dependent.
There are some methods though, that address common issues, like bug reduction or code readability.
A noneexhaustive list might be:
\begin{itemize}
  \setlength\itemsep{-0.3em}
  \item make use of programming patterns
  \item establish a clear development cycle
  \item use a bug tracker
  \item write tests
  \item make use of static analysis
  \item make use of model checking
\end{itemize}
The last two are particularly interesting because they are executed on the source code itself.
No human error possible.

Static analysis is typically part of programming languages.
One popular invariant this approach guarantees, is type safety.
But some recent programming languages like Rust\cite{} make use of additional invariants.
In the case of Rust, the ownership model is the most notable additional concept.
With strict mutability and aliasing rules, the ownership model avoids a variety of memory errors.
The price is a restricted expressiveness, that manifests in compiler warnings that are unknown to other languages.

Model checking is typically not a feature of programming languages, as it requires a mapping to a formalism.
But if such a mapping exists, desireable properties can be proven mathematically for all possible runs of the program.
A formalism that is especially suited to analyse concurrent flow of execution is Petri-nets.

In this work we will try to find a mapping between rust and Petri-nets to analyse properties of concurrency in rust programs.
We will check for deadlocks as an example property in small programs to prove the concept.

\section{Outline}
Lets see what we can expect in the following chapters.

Before we can dive into the specifics we need to establish the basics.
In the background section we will first take a closer look on \hyperlink{rel_rust}{Rust}.
%TODO: Compiler???
Next, we look at the challenges of \hyperlink{rel_para}{parallel programming} and why they are important.
This will also cover the relationship of rust with concurrency.
And the last significant topic we have to establish is \hyperlink{rel_mc}{Model Checking}.

After we understand those principles, we can move on to the approach taken.
We will again start with the specifics of \hyperlink{app_rust}{Rust}.
How can we use generic code to translate that into something we need,
and what language concepts do we need to understand for translation.
We will see that the rust language has an \hyperlink{app_mir}{intermediate representation} that simplifies our situation.
The most important part is probably the \hyperlink{app_trans}{translation} of the language features to a petri net.
It will contain a detailed description of our mapping and its limits.
The generated petri net has to be analysed by a Model Checker. 
\hyperlink{app_mc}{How this is done} und on which \hyperlink{app_test}{test programs} we will discuss next.

Finally wie will \hyperlink{results}{discuss the results} and infer \hyperlink{future}{future work} from the implementations shortcomings.
\begin{verbatim}
   Und nun ein toller Satz zur Überleitung.
\end{verbatim}